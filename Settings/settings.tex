%1234567890123456789012345678901234567890123456789012345678901234567890123456789
%_______________________________________________________________________________
% thesis.tex                                                                  
% This is the main file, which calls up preamble.tex, frontmatter.tex, and 
%thesis.bib as needed. 
%_______________________________________________________________________________


\documentclass[12pt,oneside]{sfsuthesis}  
%WHEN THESIS IS COMPLETED, CHANGE THE NEXT LINE FROM draft TO final, then the paper will be formatted according to the Graduate Division's guidelines
\usepackage[final]{Settings/MAThesisOutputFormat}
\RequirePackage{standalone}

%==========%
% biblatex %
%==========%

\usepackage[backend=biber,style=numeric]{biblatex}
\addbibresource{Bibl/thesis.bib}


%=========================%
% CUSTOM PACKAGES GO HERE %
%=========================%

% Frontmatter shortcuts; note the extra space at the end, which is unfortunately necessary:      
\newcommand\myname{John P Gallagher}
\newcommand\mytitle{AN ERDOS SIMILARITY PROBLEM IN A TOPOLOGICAL SETTING} % The graduate division requires this to be in caps.
\newcommand\mydegree{Master of Arts} % change to  Master of Science if applicable
\newcommand\myfield{Mathematics} % e.g., Mathematics
\newcommand\thismonth{May} % graduation month: May / August / December
\newcommand\thisyear{2022} % e.g., 2014

\newcommand\myadviser{Dr. Chun-Kit Lai} % Adviser
\newcommand\myadviserstitle{Associate Professor}
\newcommand\committeememberone{Dr. Emily Clader} % Committee member 1
\newcommand\committeememberonetitle{Assistant Professor} 
\newcommand\committeemembertwo{Dr. Arek Goetz} % Committee member 2
\newcommand\committeemembertwotitle{Professor}

\newcommand{\ubar}[1]{\underaccent{\bar}{#1}}
\newcommand{\N}{\mathbb{N}}
\newcommand{\Z}{\mathbb{Z}}
\newcommand{\Q}{\mathbb{Q}}
\newcommand{\R}{\mathbb{R}}
\newcommand{\F}{\mathbb{F}}
\newcommand{\C}{\mathbb{C}}
\newcommand{\X}{\mathbf{X}}
\newcommand{\Y}{\mathbf{Y}}
\newcommand{\T}{\mathcal{T}}
\newcommand{\LL}{\mathcal{L}}
\newcommand{\nullsp}{\text{null}}
\newcommand{\range}{\text{range }}
\DeclareMathOperator{\lcm}{lcm}
\DeclareMathOperator{\ima}{Im}
\DeclareMathOperator{\spn}{span}