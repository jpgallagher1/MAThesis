\section{Affine Copies and the Self Similarity Property}

%move definitions to introduction
In order to define self-similar sets, we first need to define what an affine copy of a set is.  
\begin{definition}[Affine copy]
    An \underline{affine} copy of a set $A$ is a scaled and translated set $A'$ such that for some $\lambda\neq 0, \lambda \in \R$ and $t \in \R$,  $$A' = \{\lambda a + t : a \in A\}.$$
\end{definition}
In this instance, it is a scaled and then translated copy of the set and need not be a "shape".  This gives us some flexibility when addressing different sets.  In particular let us examine fractals.

Fractals are geometric objects that have a self-similar property, and a fractional dimension.  For the scope of this paper, we will not be discussing dimension, and will instead investigate the notion of self-similarity. 

\begin{definition}[Iterated Function System]
    An iterated function system is a finite set of contraction mappings on a complete metric space.  Symbolically, we write this as, for some $N \in \N$,
    $$\{f_i:X \to X \vert i = 1,2,\dots, N\}, $$
\end{definition}

The invariant set under this iterated function system is a self similar set.

\begin{definition}[Self-Similar Set]
    A set $A$ is self-similar if it is the invariant set of an iterated function system. 
\end{definition}

Admittedly this is an abstract definition, so we come back to the Cantor set from earlier.  
\begin{claim}The Cantor set is self-similar.   
\end{claim}
\begin{proof}
    Recall the definition of the Cantor set, as the iterated removal of the middle third.      
    $$\mathcal{C} = [0,1] \setminus \bigcup_{n=0}^\infty\bigcup_{k=0}^{3^n-1}\left(\frac{3k+1}{3^{n+1}},\frac{3k+2}{3^{n+1}}\right)$$

    Here we notice that for the first removal, $n = 0$, we are left with the left and right portion of the Cantor set where 
\end{proof}
