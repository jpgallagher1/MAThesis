\chapter{Introduction}
%%%%%Re write these sections
%From the perspective of applied math, empirical results are almost always discrete observations yet the relationships that are observed are not necessarily quantized.  This means that as we infer mathematical relationships from discrete sets, there can often be a mismatch between our theoretical model, and the true relationships.  %This notion of embedded relationships is also a core area of study within pure mathematics.  
%Whether observing a pattern and trying to generalize the relationship to a broader context, or deducing a relationship from a different set of assumptions, discrete patters are intrinsically embedded in our universe. In this thesis we look to examine patters within patters.  

%From dynamical systems, and iterated function systems (IFS), the set of invariant points is an important area of study.  For example a one dimensional fractal, the Cantor set, the uncountable collection of points in the Cantor set is also the set of invariant points under an IFS.  From this perspective we can see that Cantor sets are an important set to understand from dynamical systems and iterated functions.  

Often in data work, one may ask, "Which patterns are possible to find, given a certain data set, or hypothetical relationship?"  Mathematically a similar problem can be proposed.  Which patterns, finite or infinite, exist within another collection of sets?  

If we more deeply consider patterns which infinitely repeat scaled copies of itself, this turns out to be a self-similar set.  This is very closely related to the concept of fractals as well as a core area of study within dynamical systems.  Some fractals can be generated using recursive functions.  In dynamical systems some iterated functions may have sets of invariant points.  These collections of invariant points may themselves be self-similar.  In this respect it is valuable to explore which patterns are found everywhere because they give insight into some key facets within dynamical systems as well as the nature of patterns.  

Given a specific set of points, we can formalize this notion of a scaled copy of a pattern as an \textit{affine transformation}.  We will formally define this in the next section.
% \begin{definition}[Affine copy]
%     An \underline{affine} copy of a set $A$ is a scaled and translated set $A'$ such that for some $\lambda\neq 0, \lambda \in \R$ and $t \in \R$,  $$A' = \{\lambda a + t : a \in A\}.$$
% \end{definition}

Exploring this notion a little more deeply, we can begin to investigate which patterns appear everywhere.  Informally, a set is called \textit{universal} in another collection, when every subset of the collection contains some scaled and translated copy of original pattern.  
% In the context of a measurable set:

% \begin{definition}[Measure-Universal]
%     A set $E$ is called \underline{Measure-universal} in $\R$ if for every measurable subset $S \subseteq \R$, with positive positive lebesgue measure, $\mu (S) > 0$, there exist an affine copy of $E$ such that $t+\lambda E \subseteq S,$ for some $\lambda \neq 0$ and $t \in \R$.  
% \end{definition}
  Paul Erd\H{o}s proposed a conjecture that no infinite set, is measure-universal in  the collection of sets with positive measure.  We will explore an analogous problem in a topological setting.   This also answers a question posed by Svetic\cite{Svetic}, ``Is it true that for every uncountably infinite set, $E$, of real numbers, there exists $S \subset [0,1]$ of full measure that does not contain an affine copy of $E$?"  Additionally this result can be used to prove some results in higher dimensions.  

    %Patterns in Data sets
    %Def Affine Copy
    %Def Measure-Universal
    %State Erd\H{o}s Conjecture (only need to focus on infinite sequences that decay to zero.  The rest follows)
    %History/Context for Erd\H{o}s Conjecture, 
        %Proposed Year (1974)
        %Who are main researchers? 
        %Steinhaus: Finite sets are universal (predate Erd\H{o}s)
        %Falconer(slow decay sequence, ratio test), 
        %Kolo
        %Bourgain: Faster decay sequences
        %What is current Progress?
        %Falconer(slow decay sequence, ratio test), 
        %Kolountzakis using probabilistic arguments
        %Bourgain: Faster decay sequences
        %2^{-n} is still an open question
    %Def Topological-Universal
    %Propose Analogous Theorem
    %Outline Paper Sections ahead
        %Measure ideas:
        %Topology ideas:
        %Gap Lemma
        %Main Thm
        %Multi-Dim Thm



% \begin{definition}[Topological-Universal]
%     A set $E$ is called \underline{Topological-universal} in $\R$ if every dense $G_\delta$ subset $S \subseteq X$, , there exist an affine copy of $E$ such that $t+\lambda E \subseteq S,$ for some $\lambda \neq 0$ and $t \in \R$.  
% \end{definition}
%  Any set containing an interval cannot be topologically universal.  We also have the new result that any Cantor sets is not topologically universal.  Cantor sets, which contains no interval and are uncountably infinite, are not topologically universal in the collection of dense G-delta sets.  


% If it is countable then baire category works directly.  So we need to work with 
%Indeed as we examine our world we often notice similarities we want to measure.  Using that same measure we want to look for consistency, and should we see something similar to our original observation we would expect a similar measure.  

%However in mathematics (and life), the rules we construct often lack the subtly to account for nuances. Let us na\"{i}vely consider measurement.  Suppose we have a string and want to find out its length.  After pulling the string taught along a ruler, we might see it is a few centimeters long.  Then from this collection of tools we might say that we only need two points and a ruler to be able to describe length.  In reality we have only learned of distance.  

%From that same construction though, we could just as well say, 2 points have no length at all because there is nothing between them.  In a sense, both are simultaneously true, because our definition does not address these nuances.  This motivates a few new questions: How many points do you need to add in, before we can have a length of string?  Can we use this string to measure other things?  Can we use collections of points to measure other things?  And maybe strangest of all, can collections of points have the same length as a piece of string?

%We can now go back to our notion of the universe and ask ourselves these questions again.  Suppose we change universes, does our notion of length still exist? In a different universe can we find similar copies of these collections of points? 

% Key words:\begin{itemize}
%     \item Dynamical Systems
%     \item Density and Measure are not clearly linked
%     \item Geometry
%     \item Fractals poorly defined
%     \item Self Similarity is more well defined
%     \item Cantor Set
%     \item $\R^n$ Fractals
%     \item Erd\H{o}s Proposed Conjecture with Measure Space assumptions %add in this as motivation, and include Angel and ChunKit paper
%     \item Theorem with Topological Assumptions.
%     \item Open questions
% \end{itemize}


