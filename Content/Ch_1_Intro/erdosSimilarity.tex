\section{An Erd\H{o}s Self\hyphen{Similarity} Conjecture in Measure Space}
    %History/Context for Erdos Conjecture, 
        %Proposed Year (1974)
        %Who are main researchers? 
        %Steinhaus: Finite sets are universal (predate Erdos)
        %Falconer(slow decay sequence, ratio test), 
        %Kolo
        %Bourgain: Faster decay sequences
        %What is current Progress?
        %Falconer(slow decay sequence, ratio test), 
        %Kolountzakis using probabilistic arguments
        %2^{-n} is still an open question
There is a long standing conjecture from Paul Erd\H{o}s on universal sets.  Informally the conjecture states there is no infinite set that is universal in the real number line.  This is a conjecture about which types of  patterns can exist within another set of numbers.  


We can now start by defining \textit{affine transformations} or \textit{affine copies}.
\begin{definition}[Affine copy]
    An \underline{affine copy} of a set $A \subset \R$ is a set $A'$ such that for some $ t, \lambda \in \R$ with $\lambda \neq 0$,  $$A' = \{\lambda a + t : a \in A\}.$$
\end{definition}

In this instance, a scaled and then translated copy of the set is still one dimensional.  This gives us some flexibility when addressing different sets.  This definition is used to define measure universal. 

\begin{definition}[Measure Universal]
    A set $E$ is called \underline{measure universal} in $X$ if for every subset $S \subseteq X$, with positive Lebesgue measure, $\mu (S) > 0$, there exist an affine copy of $E$ such that $E' \subseteq S,$
\end{definition}

Now we can formally, state Erd\H{o}s' conjecture as follows. 

\begin{conjecture}[The Erd\H{o}s Similarity Conjecture]\label{ErdConj}
    Let $E\subseteq \R$ be an infinite set of real numbers.  Then there is a set of real numbers $S$ of positive measure which does not contain an affine copy of $E$.  
    
    I.e.  No infinite subset of $\R$ is measure universal.  
\end{conjecture}

Paul Erd\H{o}s originally posed this question back in 1974 by building off of the work of Steinhaus.  Steinhaus\cite{Steinhaus} first posed that finite sets are universal in sets with positive measure.  In the time since Erd\H{o}s first posed this conjecture, there has been some progress.

Falconer \cite{Falconer} made a substantial progress by showing slowly decaying sequences are not measure universal.  Bourgain \cite{Bourgain} expanded on this by showing some faster decaying sequences are also not measure universal.  In particular he demonstrated that the sum-set of any three sets, cannot be measure universal.  Most recently Kolountzakis \cite{Kolo} demonstrated using probabilistic arguments that certain set with large gaps cannot be measure universal.  

Currently it is still an open question whether or not sequences that decay at the rate of $2^{-n}$ are measure universal. 


In this paper we take this idea of measure universality and put it into a topological context.  We show in Theorem \ref{theorem_positive_NW} a Cantor set with positive Newhouse thickness is not topologically universal.


% \begin{conjecture}
%     There is no infinite universal set. 
% \end{conjecture}

