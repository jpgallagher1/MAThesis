\section{An Erd\"{o}s Self-Similarity Conjecture in Measure Space}
    %History/Context for Erdos Conjecture, 
        %Proposed Year (1974)
        %Who are main researchers? 
        %Steinhaus: Finite sets are universal (predate Erdos)
        %Falconer(slow decay sequence, ratio test), 
        %Kolo
        %Bourgain: Faster decay sequences
        %What is current Progress?
        %Falconer(slow decay sequence, ratio test), 
        %Kolountzakis using probabilistic arguments
        %Bourgain: Faster decay sequences
        %2^{-n} is still an open question
There is a long standing conjecture from Paul Erd\"{o}s on universal sets.  Informally the conjecture states that there is no infinite set that is universal in the real number line.  This is a conjecture about which types of  patterns can exist within another sets of numbers.  Before we formally state the theorem, we first need to review a few definitions: affine copies, universality, and non-zero measure. 
\begin{definition}[Affine copy]
    An \underline{affine} copy of a set $A$ is a scaled and translated set $A'$ such that for some $\lambda\neq 0, \lambda \in \R$ and $t \in \R$,  $$A' = \{\lambda a + t : a \in A\}.$$
\end{definition}
In this instance, it is a scaled and then translated copy of the set is still one dimensional and therefore need not be a ``shape".  This gives us some flexibility when addressing different sets.  This definition is used to define universal. 

\begin{definition}[Measure-Universal]
    A set $E$ is called \underline{Measure-universal} in $X$ if for every subset $S \subseteq X$, with positive measure, $\mu (S) > 0$, there exist an affine copy of $E$ such that $t+\lambda E \subseteq S,$ for some $\lambda \neq 0$ and $t \in \R$.  
\end{definition}

Now we can formally, state Erd\"{o}s' conjecture as follows. 

\begin{conjecture}[The Erd\"{o}s Self-Similarity Conjecture]\label{ErdConj}
    Let $E\subseteq \R$ be an infinite set of real numbers.  Prove that there is a set of real numbers $S$ of positive measure which does not contain an affine copy of $E$.  
\end{conjecture}

Paul Erd\"{o}s first proposed this conjecture in 1978.  Originally the idea was first explored as part of 

In this paper we take this idea of universality and put it into a topological context.  We show that no cantor set is universal in the set of dense $G_\delta$ sets.  Instead of sets with positive measure we investigate the collection of dense G-delta sets.  Any finite or countable set is found to be topologically universal. 


\begin{conjecture}
    There is no infinite universal set. 
\end{conjecture}

