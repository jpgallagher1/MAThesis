\section{Affine Copies and the Self Similarity Property}

    %Patterns in Data sets
    %Def Affine Copy
    %Def Measure-Universal
    %State Erdos Conjecture
    %History/Context for Erdos Conjecture, 
        %Proposed Year
        %Who are main researchers?
        %What is current Progress?
    %Def Topological-Universal
    %Propose Analogous Theorem
    %Outline Paper Sections ahead
        %Measure ideas:
        %Topology ideas:
        %Gap Lemma
        %Main Thm
        %Multi-Dim Thm

One of the more popularly known results from math is the fractal known as hte Mandelbrot set.  In some senses fractals can be very general objects that are considered to have fractional dimensions.  This definition can be very general but by that same token, may not always capture some of the inherent geometry of some fractal objects. Some objects with fractional dimensions have self-similar properties, and some self-similar objects have fractional dimension.  For the scope of this paper, we will not be discussing dimension, and will instead investigate the notion of self-similarity. 

Recall the earlier definition of \textit{affine}.  
\begin{definition}[Affine copy]
    An \underline{affine} copy of a set $A$ is a scaled and translated set $A'$ such that for some $\lambda\neq 0, \lambda \in \R$ and $t \in \R$,  $$A' = \{\lambda a + t : a \in A\}.$$
\end{definition}

Even one dimensional objects can have this self-similar property.  In a general sense, we can take a set and dilate and translate a copy of the set.  

\begin{definition}[Dilation]
    Let $r>0$ and $a \in \R$.  The \underline{dilation} on $\R$ with ratio $r$ and center $a$ is the function $f: \R \to \R$ given by $$f(x)= rx + (1-r) a.$$
\end{definition}

Now we consider the 

\begin{definition}[Self-Similar Set]
    A set $A$ is self-similar if it is the invariant set of an iterated function system. 
\end{definition}

Admittedly this is an abstract definition, so we come back to the Cantor set from earlier.  
\begin{claim}The Cantor set is self-similar.   
\end{claim}
\begin{proof}
    Recall the definition of the Cantor set, as the iterated removal of the middle third.      
    $$\mathcal{C} = [0,1] \setminus \bigcup_{n=0}^\infty\bigcup_{k=0}^{3^n-1}\left(\frac{3k+1}{3^{n+1}},\frac{3k+2}{3^{n+1}}\right)$$

    Here we notice that for the first removal, $n = 0$, we are left with the left and right portion of the Cantor set.  Specifically the left side is a translated copy of the right side:

    $$\left\{ \left[0,\frac{1}{3}\right] \setminus \bigcup_{n=0}^\infty\bigcup_{k=0}^{3^n-1}\left(\frac{3k+1}{3^{n+1}},\frac{3k+2}{3^{n+1}}\right) \right\} + \frac{2}{3} = \left[\frac{2}{3},1\right] \setminus \bigcup_{n=0}^\infty\bigcup_{k=0}^{3^n-1}\left(\frac{3k+1}{3^{n+1}},\frac{3k+2}{3^{n+1}}\right).$$

    We note that this happens at every level where 

\end{proof}
\begin{definition}[Iterated Function System]
    An iterated function system is a finite set of contraction mappings on a complete metric space.  Symbolically, we write this as, for some $N \in \N$,
    $$\{f_i:X \to X \vert i = 1,2,\dots, N\}, $$
\end{definition}

The invariant set under this iterated function system is a self similar set.