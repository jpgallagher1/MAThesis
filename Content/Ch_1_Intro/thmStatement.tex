\section{An Analogous Theorem in a Topological Setting}

\begin{definition}[Topological-Universal]
    A set $E$ is called \underline{Topological-universal} in $X$ if for every subset $S \subseteq X$, with positive measure, $\mu (S) > 0$, there exist an affine copy of $E$ such that $t+\lambda E \subseteq S,$ for some $\lambda \neq 0$ and $t \in \R$.  
\end{definition}
 Any set containing an interval cannot be topologically universal.  We also have the new result that any Cantor sets is not topologically universal.  Cantor sets, which contains no interval and are uncountably infinite, are not topologically universal in the collection of dense G-delta sets.  
