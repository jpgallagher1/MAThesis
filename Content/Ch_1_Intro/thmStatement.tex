\section{An Analogous Theorem in a Topological Setting}

In a non-rigorous exploration of the real number line, one might assume that patterns which appear everywhere should have positive measure.  However density and measure are not intrinsically linked.  Indeed it is possible to have uncountable dense sets with measure zero and to have sets with full measure that are nowhere dense.  

Borrowing the concept of $G_\delta$ sets from topology, we can explore an extension of the Erd\H{o}s similarity problem.  Instead of exploring sets with positive measure we can explore dense $G_\delta$ sets.  This is the countable intersection of open sets that are also dense.  We will use these dense $G_\delta$ sets to define topological-universal. 

\begin{definition}[Topological-Universal]
    A set $E$ is called \underline{Topological-universal} in $\R$ if for every dense $G_\delta$ subset $S \subseteq \R$, there exist an affine copy of $E$ such that $t+\lambda E \subseteq S,$ for some $\lambda \neq 0$ and $t \in \R$.  
\end{definition}

As stated above, a dense $G_\delta$ set can have measure zero.  This also means that there is not a direct relationship between topological universal and measure universal.  We make two observations on this fact: a set with an interior cannot have be topologically universal, by the Baire Category Theorem all countable sets are topologically universal.   

This motivates the question, Is a nowhere dense set, with an empty interior topologically universal?  Cantor sets for example, have an empty interior and are nowhere dense.  

In chapter 2 we review background definitions and theorems in measure theory and topology.  We also give several examples to demonstrate some of the nuances of these facts.  In chapter 3 we define Newhouse Thickness, prove the Gap Lemma and prove our main result theorem \ref{theorem_positive_NW}.

\begin{theorem*}
Let $J$ be a cantor set with positive Newhouse thickness.  Then $J$ is not topologically universal.
\end{theorem*}

These results can also be extrapolated into $\R^d$ by defining projective Newhouse thickness.  Finally in chapter 4 we conclude with some open questions and remarks.  
% From these

% We have the new result that any Cantor sets with positive Newhouse thickness is not topologically universal.  We also conjecture that all cantor sets are not topologically universal.   

