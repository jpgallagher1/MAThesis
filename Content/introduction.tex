\chapter{Introduction}

From the perspective of applied math, empirical results are almost always discrete observations yet the relationships that are observed are not necessarily quantized.  This means that as we infer mathematical relationships from discrete sets, there can often be a mismatch between our theoretical model, and the true relationships.  This notion of embedded relationships is also a core area of study within pure mathematics.  Whether observing a pattern and trying to generalize the relationship to a broader context, or deducing a relationship from a different set of assumptions, discrete patters are intrinsically embedded in our universe.  Indeed as we examine our world we often notice similarities we want to measure.  Using that same measure we want to look for consistency, and should we see something similar to our original observation we would expect a similar measure.  

However in mathematics (and life), the rules we construct often lack the subtly to account for nuances. Let us na\"{i}vely consider measurement.  Suppose we have a string and want to find out its length.  After pulling the string taught along a ruler, we might see it is a few centimeters long.  Then from this collection of tools we might say that we only need two points and a ruler to be able to describe length.  In reality we have only learned of distance.  

From that same construction though, we could just as well say, 2 points have no length at all because there is nothing between them.  In a sense, both are simultaneously true, because our definition does not address these nuances.  This motivates a few new questions: How many points do you need to add in, before we can have a length of string?  Can we use this string to measure other things?  Can we use collections of points to measure other things?  And maybe strangest of all, can collections of points have the same length as a piece of string?

We can now go back to our notion of the universe and ask ourselves these questions again.  Suppose we change universes, does our notion of length still exist? In a different universe can we find similar copies of these collections of points? 

Key words:\begin{itemize}
    \item Dynamical Systems
    \item Density and Measure are not clearly linked
    \item Geometry
    \item Fractals poorly defined
    \item Self Similarity is more well defined
    \item Cantor Set
    \item $\R^n$ Fractals
    \item Erd\"{o}s Proposed Conjecture with Measure Space assumptions
    \item Theorem with Topological Assumptions.
    \item Open questions
\end{itemize}

%\section{Introduction to the Introduction}



%\subsection{Introduction to the Introduction to the Introduction}


