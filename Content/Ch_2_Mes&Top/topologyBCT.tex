\section{Topological and The Baire Category Theorem}


We also need to define \textit{dense}.  There are many equivalent definitions of dense.  We will use the following definition so that we can continue to develop the intuion around intervals and interiors. 

\begin{definition}[Dense]
    A set $S$ is called dense in $X$ if for every $x \in X$, every neighborhood $U$ of $x$ intersects $A$.  
\end{definition}


In a similar fashion to the definitions of \textit{countable} and \textit{uncountable}, the oppsite of \textit{dense} is \textit{nowhere dense}.

\begin{definition}[Nowhere Dense]  Let $X$ be a topological space.  A subset $B \subseteq X$ of a topological space is called \textit{nowhere dense} in $X$ if its closure has an empty interior.  That is to say, $B$ is \textit{nowhere dense} in $X$ if for each open set $U\subseteq X$, $B\cap U$ is not dense in $U$.      
\end{definition}



This allows us to now explore the differences between density and measure.  As stated earlier, topological size and measure theoretic size are not necessarily related.  What do we mean by topologically Large? Uncountable and dense.  Similarly what do we mean by measure theoretically large?  Non-zero measure.  After reviewing t It is helpful to define the opposite of topologically large, namely meager sets.



\begin{definition}[Meager]  A subset $C \subseteq X$ of a topological space is called \textit{meager} in $X$ if it is the countable union of nowhere-dense subsets of $X$.    
\end{definition}
Now we look at our example.  
\begin{example}A measure theoretically large set is not necessarily topologically large.

Consider the interval $[0,1]$ and for all positive integers $a,n \in \N$ remove the intervals $(\frac{a}{2^n} - \frac{1}{2^{n+1}},\frac{a}{2^n} + \frac{1}{2^{n+1}})$.  Notice that the intervals are a geometric series and for each $n$ add up to at most $\frac{1}{2^{n+1}}$.  Therefore the set 
$$[0,1] \setminus \bigcup_{a,n \in \N} \left(\frac{a}{2^n} - \frac{1}{2^{n+1}},\frac{a}{2^n} + \frac{1}{2^{n+1}}\right),$$
is closed, has an empty interior, and is of positive measure.  
\end{example}

Next we will define dense $G_\delta$ sets, as well as some useful examples.  
\begin{definition}[G-Delta Set]
    A $G_\delta$ set is the countable intersection of open sets.  Namely, let $O_i \subset X$ for $i \in \N$ be a collection of open sets of $X$.  Then 
    $\bigcap_{n=1}^\infty O_i,$ is a $G_\delta$ set.  
\end{definition}

\begin{example}
    The irrational numbers are a $G_\delta$ set.  Consider the following construction of the set of irrational numbers:
    $$\R \setminus \Q = \bigcap_{q \in \Q}\R \setminus \{q\}.$$
\end{example}
Notice that each $\R\setminus \{q\} = (-\infty, q) \cup (q, \infty)$ is an open subset of $\R$.  Furthermore, rational numbers are countable.  Therefore the intersection of these sets are a $G_\delta$ set.  Moreover, in this instance it is a dense $G_\delta$ set.  We will study these objects further.  

Lastly we remark that there is an analogous set which is the countable union of closed sets.
\begin{definition}[F$_\sigma$ Set]
    An $F_\sigma$ set is the countable union of closed sets.  This is equivalent to the compliment of a G-delta is an F-sigma set.  
\end{definition}

A key theorem that links analysis to set theory is the Baire Category Theorem.  This also establishes a link to understanding certain types of topological sets.  
\begin{theorem}[Baire Category Theorem]\cite{Munkres}
    The countable intersection of open dense sets is dense.  
\end{theorem}

Within the study of measure theory it can sometimes be unclear if a set is dense in another set.  For example consider the following set: 
$$\R^2 \setminus \{(x,y): y=mx + b, \text{where $m, b \in \Q$.}\}$$
Notice that this can also be written as $$\bigcap_{m,b \in \Q} \R^2 \setminus \{(x,y): y=mx+b\}, $$ which is the plane, but removing all lines with rational coefficients, and rational intercepts.  