\section{Cantor Sets and Other Measure Theoretic and Topological Examples}
We begin this section with a special $F_\sigma$ set.  The Cantor set is defined by taking the interval $[0,1]$ and then iteratively removing the open interval containing the middle third, from the previous level.  As such it is the countable intersection of closed sets.  Formally this can be written as follows.
\begin{definition}[Cantor Set]
    The Cantor set $\mathcal{C}$, written as the successive removal of each middle third removed from the previous level is 
    $$\mathcal{C} = [0,1] \setminus \bigcup_{n=0}^\infty\bigcup_{k=0}^{3^n-1}\left(\frac{3k+1}{3^{n+1}},\frac{3k+2}{3^{n+1}}\right)$$
\end{definition}

As an $F_\sigma$ set defined on a closed interval with iteratively removed open intervals we see the middle third Cantor set can also be described with the following three properties.
\begin{definition}[Cantor Set]
    \textit{Cantor sets} are compact, perfect sets, and totally disconnected sets.      
\end{definition}

As a quick reminder to the defitions:
\begin{definition}[Perfect]
    A \textit{perfect} set is a closed set that contains no isolated points.
\end{definition}
\begin{definition}[Totally Disconnected]
    A set is \textit{totally disconnected} if the only connected components are single points. 
\end{definition}


An equivalent formulation of the Cantor set, is the decimal expansion of all numbers in $[0,1]$ in base $3$, omitting any representation with a $1$. This can be a useful tool for thinking through some examples and counter-examples.
\begin{example}[Decimal Expansion Cantor Set]
    $$\mathcal{C}  = \{ x \in [0,1]: \text{ $x$ has a ternary expansion containing no $1$'s.}\}$$
\end{example}
Here we notice that although $1/3 \in \mathcal{C} $ can be written as $0.1$ using the trinary expansion, it also has another representation as $1/3 = 0.0\overline{2}$  This would be the included representation in the Cantor set.  We take a moment to acknowledge that numbers may not have unique representations, where one may be excluded but the other included.  

Earlier we defined nowhere dense. Here we see that the Cantor set is an example of a nowhere dense set.  
% \begin{definition}[Nowhere dense]
%     A set $A \subseteq X$ is called \textit{nowhere dense} if its closure $\overline{A}$ has an empty interior.  Equivalently the set $A$ is \textit{nowhere dense} if $A$ is not not dense in any subset $U$ of $X$.  
% \end{definition}
\begin{claim}The Cantor set is \textit{nowhere dense}.  
\end{claim}  
\begin{proof}
    Let $\mcC$ be the middle third Cantor set.  Notice that $[0,1]\setminus \mcC$ is a set of open intervals: 
    $$ [0,1] \setminus \mathcal{C} =\bigcup_{n=0}^\infty\bigcup_{k=0}^{3^n-1}\left(\frac{3k+1}{3^{n+1}},\frac{3k+2}{3^{n+1}}\right). $$
    Therefore $\mcC$ is the countable intersection of closed intervals, and itself is closed.  
    
    Notice given some radius $r$, there exists some number $t$, such that $0<t<r$ and $t$ has a 1 in its ternary expansion.  So if we consider any point $c \in \mcC$, then an open ball of radius $r$ centered at $c$ then $B_r(c)$ is $(c-r, c+r)$ in ternary, necessarily contains a number containing a $1$.  Therefore $B_r(c) \not\subseteq \mcC$, and $\mcC$ has an empty interior.  Finally we conclude because $\mcC$ is closed and has an empty interior, $\mcC$ is nowhere dense.      
\end{proof}
Beyond the middle third Cantor set, we can generalize these in a few different ways.  Changing the middle interval, having a few different interval widths, series of interval widths. 
%\begin{definition}{Generalized Cantor's Set}
%    Middle Third Cantor Set
%\end{definition}

\begin{figure}
    \begin{center}
    
\begin{tikzpicture}
%\draw (-1,.5) -- (28, .5);
% \draw [(-)](9,0) -- (18, 0);
%\draw (-1,0) -- (28, 0);
%\draw (-1,-.5) -- (28, -.5);
%\draw (-1,-1) -- (28, -1);
%\draw (-1,-1.5) -- (28, -1.5);
\draw [line width = 1.5mm] (0,.5) -- (9, .5);
\draw [decoration=Cantor set,line width =1.5mm] decorate{ (0,0) -- (9,0)};
% \draw [(-)](9,0) -- (18, 0);
\draw [decoration=Cantor set,line width =1.5mm] decorate{ decorate{ (0,-.5) -- (9,-.5) }};
\draw [(-)](3,0) -- (6, 0);
% \draw [(-)](21,-0.5) -- (24, -0.5);
\draw [decoration=Cantor set,line width =1.5mm]decorate{ decorate{ decorate{ (0,-1) -- (9,-1) }}};
\draw [(-)](1,-0.5) -- (2, -0.5);
\draw [(-)](7,--0.5) -- (8, --0.5);
% \draw [(-)](19,-1) -- (20, -1);
% \draw [(-)](25,-1) -- (26, -1);
\draw [decoration=Cantor set,line width =1.5mm] decorate{ decorate{ decorate{ decorate{ (0,-1.5) -- (9,-1.5) }}}};
\draw [(-)](1/3,-1) -- (2/3, -1);
\draw [(-)](7/3,-1) -- (8/3, -1);
\draw [(-)](19/3,-1) -- (20/3, -1);
\draw [(-)](25/3,-1) -- (26/3, -1);
\end{tikzpicture}
\end{center}
\caption{The open middle intervals are recursively removed.  This removes the interior of the set, while still leaving the original set closed.  }
    \label{fig:my_label}
\end{figure}


\begin{example}[A measure theoretically large set is not necessarily topologically large.]
    The Smith-Volterra-Cantor set is formed in a similar manor to the middle-third Cantor set.  Starting with the closed interval, remove the middle fourth recursively.  At each level, $2^{n-1}$ intervals of length $1/4^n$ are removed.  The total length of these removed intervals are $$\sum_{n=0}^\infty \frac{2^n}{2^{2n+2}} = \frac{1}{2}.$$
    Therefore the Smith-Volterra-Cantor set has measure $1 - 1/2 = 1/2.$
    This closed set still has an empty interior and is therefore nowhere dense but still has positive measure. 
%Consider the interval $[0,1]$ and for all positive integers $n \in \N$ remove the intervals $(\frac{1}{2^n} - \frac{1}{2^{n+1}},\frac{a}{2^n} + \frac{1}{2^{n+1}})$.  Notice that the intervals are a geometric series and for each $n$ add up to at most $\frac{1}{2^{n+1}}$.  Therefore the set 
%$$[0,1] \setminus \bigcup_{a,n \in \N} \left(\frac{a}{2^n} - \frac{1}{2^{n+1}},\frac{a}{2^n} + \frac{1}{2^{n+1}}\right),$$
%is closed, has an empty interior, and is of positive measure.  
\end{example}

\begin{example}[A dense, uncountable set of measure zero]
As stated above, the middle third Cantor set has measure $0$.  Let $\mcC_q$ denote a cantor set translated by a rational number $q \in \Q$.  This gluing of sets is dense but still measure zero.  This is formed from closed sets. 
$$\left\vert \bigcup_{q \in \Q} \mcC_q \right\vert \leq  \sum_{q \in \Q}\left\vert\mcC_q\right\vert = 0.$$
\end{example}
