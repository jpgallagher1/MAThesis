\section{Generalizing into Higher Dimensions}
%higher dimension, affine copy will be different in a few ways: \lambda, can be an orthogonal transformation, or scalar, t can be a vector.

Consider a compact set $J$ in $\R^d$. An affine copy of $J$ is $\R^d$ is the set 
$$
t+\delta O (J)
$$
where $t\in \R^d$, $\delta\ne 0$ and $O$ is an orthogonal transformation. We say that $J$ is universal is the collection of dense $G_{\delta}$ sets if any dense $G_{\delta}$ set contains an affine copy of $J$. 
%Falconer and Yavicoli generalize newhouse thickness to higher dimension, but their definition of thickness in higher dimension has to have a path connected component.  
\begin{theorem}
    If $X\subset \R^d$ contains a path connected component, then $X$ is not universal in the set of dense $G_\delta$ sets of $\R^d$.  
\end{theorem}
\begin{proof}
    Remove the plane $$\R^d \setminus \bigcup_{i=1}^d \bigcup_{r \in \Q} \{X_i = r \}.$$
    Consider some affine copy of $X$ with a path $L$.  Then $X'$ will contain an affine copy of the path $L'$.  The projection of $L'$ onto the coordinate axises will be non-degenerate on some interval for at least one of the axises.  Call this the $i$-th axis.  This interval will contain a rational number $r$.  Therefore $L'$ will intersect with the coordinate plane, $X_i = r$ 
    In other words this dense G-delta set omits at least one point and cannot contain affine copy of $X$.  
\end{proof}
 We can still consider sets that contain no connected component.  Cantor dust contains no connected component.  We introduce the notion of \textit{projective Newhouse Thickness}, and use it to generalize our one-dimensional results into $\R^d$.  
\begin{definition}[Positive Projective Newhouse Thickness]
    We say a $J$ set has {\it positive projective Newhouse thickness} if for all $O \in O(d)$ 
$$
 \tau(P_xO(J)) > 0
$$
where $P_x$ is the orthogonal projection to the $x$-axis and $O(d)$ is the orthogonal group consisting of all orthogonal transformations in $R^d$.  
\end{definition}

We note that the projection some Cantor dust set maybe an interval.  If that is the case we say the Newhouse Thickness of the projection is equal to infinity.  

%remark or compute projective newhouse thickness.  

Using our one dimensional result, we can generalize 
\begin{theorem}Let $J \subset \R^d$ be a compact set such that it has a positive projective Newhouse thickness. Then $J$ is not topologically-universal in $\R^d$.  
\end{theorem}

\begin{proof}  Suppose we have a compact set $J$ in $\R^d$ such that it has a positive projective Newhouse thickness.  By Theorem \ref{theorem_positive_NW}, there exists a dense $G_{\delta}$ set $G_1$ in ${\mathbb R}^1$ such that $G_1$ does not contain any  affine copy of $P_x O(J)$. Now, we consider 
$$
G = G_1\underbrace{\times \dots \times}_\text{d-times} G_1. 
$$
Then $G$ is a dense $G_{\delta}$ set in ${\mathbb R}^d$. We claim that there is no affine copy of $J$ in $G$. To justify the claim. Suppose $G$ contains an affine copy of $J$ such that $t+\delta J\subset G$.  Then we take projection and we obtain that
$$
P_x(t)+ \delta P_x (J) \subset G_1
$$
which is a contradiction. Hence, the claim is true and the proof is complete. 



%From this projection we can construct a cantor set, $K$ such that $K$ is not fully contained in the gaps of $\proj_x J$ and  $\tau(\proj_x J)\tau(K)>1$.  Then from this cantor set $K$ we construct a dense $F_\sigma$ set $X$ such that 
%$$X = \bigcup_{n\in \Z} \bigcup_{l \in \Z} N^n (K +l).$$

%Then by our first theorem 

\end{proof}