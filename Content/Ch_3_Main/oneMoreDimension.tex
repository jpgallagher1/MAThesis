\section{Generalizing into Higher Dimensions}
%higher dimension, affine copy will be different in a few ways: \lambda, can be an orthogonal transformation, or scalar, t can be a vector.

Consider a compact set $J$ in $\R^d$. An affine copy of $J$ in $\R^d$ is the set 
$$
t+\delta O (J)
$$
where $t\in \R^d$, $\delta\ne 0$ and $O$ is an orthogonal transformation. We say that $J$ is universal in the collection of dense $G_{\delta}$ sets if any dense $G_{\delta}$ set contains an affine copy of $J$. 
%Falconer and Yavicoli generalize newhouse thickness to higher dimension, but their definition of thickness in higher dimension has to have a path connected component.  
\begin{theorem}
    If $X\subset \R^d$ contains a path connected component, then $X$ is not universal in the set of dense $G_\delta$ sets of $\R^d$.  
\end{theorem}
\begin{proof}
    Remove the plane 
    $$
    \R^d \setminus \bigcup_{i=1}^d \bigcup_{r \in \Q} \{X_i = r \}.
    $$
    This is clearly a dense $G_{\delta}$ set. 
    Consider any affine copy of $X$ must contain a path $L$.   The projection of $L$ onto the coordinate axes will be non-degenerate on some interval for at least one of the axes.  Call this the $i$-th axis.  This interval will contain a rational number $r$.  Therefore $L$ will intersect with the coordinate plane, $X_i = r$ 
    In other words this dense $G_{\delta}$ set omits at least one point and cannot contain affine copy of $X$.  
\end{proof}
 We can still consider sets that contain no connected component.  \underline{Cantor dust} contains no connected component.  
 \begin{definition}[Cantor Dust]
 \underline{Cantor dust} is the Cartesian product of two cantor sets.    
 \end{definition}
 
 We introduce the notion of \underline{projective Newhouse Thickness}, and use it to generalize our one-dimensional results into $\R^d$.  
\begin{definition}[Positive Projective Newhouse Thickness]
    We say a set $J$ has \underline{positive projective Newhouse thickness} if for all ${\mathcal O} \in O(d)$ 
$$
 \tau(P_x{\mathcal O}(J)) > 0
$$
where $P_x$ is the orthogonal projection to the $x$-axis and $O(d)$ is the orthogonal group consisting of all orthogonal transformations in ${\mathbb R}^d$.  
\end{definition}

We note that the projection some Cantor dust set maybe an interval.  If that is the case we say the projective Newhouse Thickness is equal to infinity.  

%remark or compute projective newhouse thickness.  

Using our one dimensional result, we can generalize 
\begin{theorem}Suppose $J \subset \R^d$ has positive projective Newhouse thickness and contains no isolated points. Then $J$ is not topologically universal in $\R^d$. 
\end{theorem}

\begin{proof}  Suppose we have a set $J$ in $\R^d$ that has a positive projective Newhouse thickness. In Theorem \ref{theorem_positive_NW}, we see that for all Cantor sets $J'$ with Newhouse thickness at least $\epsilon$, there exists a dense $G_{\delta}$ set that omits all affine copies of $J'$. Hence, if we take $\epsilon = 1/n$,  there exists $G_n$, a dense $G_{\delta}$ set in ${\mathbb R}^1$ such that $G_n$ omits all affine copies of all Cantor set $J'$s, with $$\tau(J') \geq \frac{1}{n}.$$
    % $P_x O(J)$. 
Now, from these $G_n$ we construct
$$
G = \bigcap_{n=1}^\infty G_n\underbrace{\times \dots \times}_\text{d times}G_n. 
$$
Each $G_n\underbrace{\times \dots \times}_\text{d times}G_n $ is a dense $G_\delta$ set in ${\mathbb R}^d$ and therefore $G$ is also a dense $G_\delta$ set by the Baire Category theorem.  

Now all that remains is to prove that $G$ has no affine copy of $J$.   Assume to the contrary that $G$ contains an affine copy of $J$ and denote it by $t+\lambda \mathcal{O}J$. Note that the assumption implies that 
$$
\tau(P_x(t) + \lambda P_x\mathcal{O}(J))> 0.
$$
Hence, there exists some $n$ so that the above thickness is at least $\frac{1}{n}.$ By our construction of $G_n$, $G_n$ has no affine copy of $P_x\mathcal{O}(J)$.  But if $t + \lambda \mathcal{O}J \subset G = \bigcap_{n=1}^\infty G_n \times \dots \times G_n,$ then $t+\lambda \mathcal{O}(J) \subset G_n \times \dots \times G_n$ and it implies that the projection $P_x(t) + \lambda P_x\mathcal{O}(J) \subset G_n$, which is a contradiction.  Therefore our assumption is false, and $G$ does not contain an affine copy of $J$.  \end{proof}