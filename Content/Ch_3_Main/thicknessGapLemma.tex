\section{Positive Newhouse Thickness}
Cantor sets contain important invariant structures such as Hausdorff dimension, thickness, and denseness.  We will investigate thickness and denseness.  First we will define the gaps and bounded gaps of Cantor sets in order to construct and define Newhouse thickness. 
\begin{definition}[Gap]
    Let $K$ be some Cantor set.  A \underline{gap} of $K$ is a connected components of $\R \setminus K$.      
\end{definition}  Informally, the gaps are the intervals surrounding the points of the Cantor set.  Some of the lengths of these intervals are bounded some are not.  In the example of the middle third Cantor set, the unbounded gaps would be $(-\infty, 0)$ and $(1, \infty)$.  

\begin{definition}[Bounded Gap]
    Let $K$ be a Cantor set.  A \underline{bounded gap} is a bounded connected component of $\R \setminus K$.      
\end{definition}

Using these two notions we will define the \underline{bridge} of $C$ of Cantor set $K$.  
\begin{definition}[bridge]\cite{palis&takens}
    Let $U$ be any bounded gap and $u$ be a boundary point of $U$.  Note that $u \in K$.  The \underline{bridge}    
\end{definition}

\begin{definition}[Newhouse Thickness]\cite{palis&takens}
    Let $K$ be some set 
    
\end{definition}


%include examples of calculating newhouse thickness
%Cantor set Middle Third
%N Digit
%Comment on removing left interval doesn't work well.  Thin left interval versus Thick Right interval
%Also note that all self-similar Cantor sets in \R^1 have positive newhouse thickness
\section{The Gap Lemma}
\begin{lemma}{The Gap Lemma}\cite{palis&takens}
    Let $K_1, K_2, \subset \R$ be Cantor sets with thickness $\tau_1$ and  $\tau_2$.  If $\tau_1 \cdot \tau_2 >1,$ then one of the following three alternatives occurs: $K_1$ is contained the gap of $K_2$; $K_2$ is contained in the gap of $K_1$; $K_1 \cap K_2 \neq \emptyset.$
\end{lemma}
First we will assume for the sake of contradiction that the opposite is true.  These assumptions lead to claim, which under these assumptions must be true.  Then that claim leads to a contradiction which means our original assumption must be false.  
\begin{proof}
    Let $K_1, K_2$ be two Cantor sets with thickness $\tau_1, \tau_2$ respectively and assume that $K_1$ is not contained in the gap of $K_2$ and $K_2$ is not contained in the gap of $K_1$.  

    Assume for the sake of contradiction that $K_1 \cap K_2 = \emptyset$.  Consider the bounded gaps $U_1 \subset K_1^c$ and $U_2 \subset K_2^c$.  We call $(U_1, U_2)$ a gap-pair if $U_1$ contains exactly one boundary point of $U_2$ and $U_2$ contains exactly one point of $U_1$.  

    By assumption we know that $K_1, K_2$ are not contained in the other's gaps. Therefore there exists some gap-pair $(U_1, U_2).$

    \begin{claim}
        If $\tau_1\tau_2 > 1$ then from the interval $U_1$ (or for that matter $U_2$) we can construct another sub-interval $U_1'$ such that $l(U_1') < l(U_1)$ (or similarly $U_2'$ such that $l(U_2')<l(U_2)).$
    \end{claim}
    %Proof of claim
    Notice that $(U_1',U_2)$ is still a gap-pair, as is $(U_1,U_2')$.
    
    Using this construction we can create a sequence of gap-pairs $(U_1^{(i)}, U_2^{(j)})$.  Notice that because it is a summable compact cantor set, the $U_1^{(i)}$ and $U_2^{(j)}$ are compact.  Moreover the sums of the lengths $$\sum_i^\infty l(U_1^{(i)}) < \infty,$$ and therefore $l((U_1^{(i)} \to 0$ as $i \to \infty$.   From this we construction we have a sequence of gap-pairs such that as $i \to \infty$, $l(U_1^{(i)}) \to 0$ and similarly as $j \to \infty$, $l(U_2^{(j)}) \to 0$.  
    
    Without loss of generality, we can just use the same indexing the gap pairs, $(U_1^{(i)}, U_2^{(i)})$.  If we pick a sequence of points, $q_{i} \in U_1^{(i)}$ then by the Bolzano Weierstrass theorem, there is a convergent subsequence $q_{i_k} \to q$.  Notice that $U_1^{(i)} = (a_i, b_i)$ is not fully contained in the gap of $K_2$.  Moreover because these intervals are compact and nested, we know that $a_{i_k} - b_{i_k} \to 0$ which implies that $q \in K_2$.  

    Because this construction is symmetric, the same argument applies to $q_{i} \in U_2^{(i)}$ and so $q_{i_k} \to q$ implies that $q \in K_1 \cap K_2$.  

    We want to use this technique to demonstrate that $K_1 \cap K_2 \neq \emptyset.$ Let $C_j^l, C_j^r$ denote the bridges of $K_j$ for $j = 1,2$.  Returning to our original assumptions, $\tau_1 \cdot \tau_2 > 1$ and therefore $$\frac{l(C_1)}{l(U_2)} \cdot \frac{l(C_2)}{l(U_1)} > 1.$$  
    
    From our construction, the right endpoint of $U_2$ is in $C_1^r$ or the left endpoint of $U_1$ is in $C_2^l$ or both.  
    In the case that $q \in U_2$ is the right endpoint, then $q \in K_1$ and $q \in K_2$ and we are done.  
    If $q \not\in K_1,$ then $q \in U_1'$, the gap of $K_1$ where $l(U_1') < l(U_1)$ and $(U_1', U_2)$ is the gap pair we need.  

\end{proof}