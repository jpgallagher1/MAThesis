\section{Positive Newhouse Thickness}
Cantor sets contain important invariant structures such as Hausdorff dimension, thickness, and denseness.  We will investigate thickness and denseness.  First we will define the gaps and bounded gaps of Cantor sets in order to construct and define Newhouse thickness. 
\begin{definition}[Gap]
    Let $K$ be some Cantor set.  A \textit{gap} of $K$ is a connected components of $\R \setminus K$.      
\end{definition}  Informally, the gaps are the intervals surrounding the points of the Cantor set.  Some of the lengths of these intervals are bounded some are not.  In the example of the middle third Cantor set, the unbounded gaps would be $(-\infty, 0)$ and $(1, \infty)$.  

\begin{definition}[Bounded Gap]
    Let $K$ be a Cantor set.  A \textit{bounded gap} is a bounded connected component of $\R \setminus K$.      
\end{definition}

Using these two notions we will define the \textit{bridge} of $C$ of Cantor set $K$.  
\begin{definition}[Bridge]\cite{palis&takens}
    Let $K$ be some cantor set and $U$ be a bounded gap of $K$ with boundary point $u$.  The \textit{bridge} $C$ of $K$ at $u$ is the maximal interval in $\R$ such that:
    \begin{itemize}
        \item $u$ is a boundary point of $C$
        \item $C$ contains no point of a gap $U'$ whose length $\ell(U') \geq \ell(U)$..
    \end{itemize}
\end{definition}

For clarity the picture below shows that there may be smaller bounded gaps contained in $C$.  

\vspace*{0.25cm}
% \begin{figure}
% \centering
\begin{tikzpicture}
    \draw (-6.5,0) -- (6.5,0) ;
    \draw[(-)] (-5,0)--(-3,0);
    \draw[very thick] (-5,0) --  (-3,0) node[midway, above]{$U$} node[below]{$u$};
    \draw[(-)] (-2,-1)--(-1.25,-1)node[midway, below]{$U_1$};
    %\draw[very thick] (0.92,0) -- (1.92,0);
    \draw[(-)] (0.5,-1) -- (1.75,-1)node[midway, below]{$U_2$};
    \draw[(-)] (3,0)--(6,0);
    \draw[very thick] (3,0) -- (6,0)node[midway, above]{$U'$};
    \draw[below,{[-]}]  (-3,-0.75) -- (3,-0.75) node[midway, above]{$C$};
    \draw[very thick]  (-3,-0.75) -- (3,-0.75);
\end{tikzpicture}
% \end{figure}

We use this notion to define the \textit{Newhouse Thickness}.  Intuitively the thickness of a Cantor set can be thought of as a the infimimum of ratios between the bounded gaps and the bridges.  
\begin{definition}[Newhouse Thickness \cite{palis&takens}]
    The \textit{Newhouse Thickness or thickness of $K$ at $u$} is defined as
    $$\tau(K, u) = \frac{\ell(C)}{\ell(U)}.$$
    Moreover for $\mathcal{U} = \{ \text{set of all boundary points of bounded gaps}\}$, the thickness of the entire Cantor set is 
    $$\tau(K) = \inf_{u\in \mathcal{U}} \tau(K, u) = \inf_{u\in \mathcal{U}}\frac{\ell(C)}{\ell(U)}$$
\end{definition}
This helps us develop a language to talk about the relative sizes of Cantor sets that is slightly separated from the notion of measure.  Importantly it also helps us for a sufficient condition for our main theorem.

Here we will calculate a few examples of Newhouse Thickness.  Recall the middle third Cantor set.  
\begin{example}[Newhouse Thickness of the Middle\hyphen{third} Cantor Set]
    Let $K$ be the middle third cantor set.  Then the Newhouse thickness is the infimum of the ratio between gaps and bridges.  Here we notice that every bounded gap is one third the previous bridge.  Therefore the Newhouse thickness of the set is 
    $$\tau(K) = \inf_{u\in K}\frac{\ell(C)}{\ell(U)} = \frac{\frac{1}{3}}{\frac{1}{3}} = 1.$$
\end{example}

\begin{example}[Newhouse Thickness of the $N$\hyphen{digit} Cantor Set]
    Let $K$ be an $N$\hyphen{digit} Cantor set.  Each gap at the $n$\hyphen{th} level is of has length $N^{-n}$.  We take a moment to note that the location of the gap matters because it affects the thickness.  If we assume that for $2 \leq j \leq n-1$, the $j$\hyphen{th} digit is removed then we end up with the following cantor set. 
    % Damn I drew the wrong pictures... should have just copied from above.
    % \begin{tikzpicture}
    %     \draw (-7,-0.75) -- (7,-0.75) ; %total length 14
    %     \draw (-7,-1.75) -- (7,-1.75) ; %total length 12
    %     \draw[{[-]}] (-6,-0.75)--(-2,-0.75); %sub length 4
    %     \draw[ultra thick] (-6,-0.75) --  (-2,-0.75) node[midway, above]{$C_i$} node[above]{$j_i$}; %left side
    %     \draw[(-)] (-2,-1)--(-1,-1)node[midway, below]{$U_i$}; %bdd gap
    %     \draw[(-] (5,-1)--(7,-1)node[midway, below]{$U_{i+1}$}; %bdd gap
        
    %     \draw[{[-]}]  (-1,-0.75) -- (5,-0.75) node[midway, above]{$C_{i+1}$};%sublength 7
    %     \draw[ultra thick] (-1,-0.75) -- (5,-0.75); %right side

    %     %left, scaled by 1/4
    %     \draw[{[-]}] (-6,-1.75) -- (-5,-1.75);%node[midway, below]{$C_{i+2}$};
    %     \draw[very thick] (-6,-1.75) -- (-5,-1.75);
    %     \draw[(-)] (-5,-2)-- (-4.75,-2);%node[midway, below]{$U_{i+1}$};
    %     \draw[{[-]}] (-4.75,-1.75)--(-2,-1.75) ;%node[midway, below]{$C_{i+3}$};
    %     \draw[very thick] (-4.75,-1.75)--(-2,-1.75);
    %     \draw (-5.5,-2.25)node{$\vdots$};
    %     \draw (-3.25,-2.25)node{$\vdots$};    
        
    %     %right scaled by 1/7 +5
    %     \draw[{[-]}]  (-1,-1.75) -- (6/7,-1.75);% node[midway, below]{$C_{i+4}$};
    %     \draw[very thick] (-1,-1.75) -- (6/7,-1.75);
    %     \draw[(-)] (6/7,-2)-- (6/7+7/12,-2) ;%node[midway, below]{$U_{i+2}$};
    %     \draw[{[-]}] (6/7+7/12,-1.75)--(5,-1.75) ;%node[midway, below]{$C_{i+5}$};
    %     \draw[very thick] (6/7+7/12,-1.75)--(5,-1.75);
    %     \draw (-1/14,-2.25)node{$\vdots$}; 
    %     \draw (2.5+3/7 +7/24,-2.25)node{$\vdots$}; 
    % \end{tikzpicture}

%the right picture...
    \begin{tikzpicture}
        \draw (-6.5,0) -- (6.5,0) ;
        \draw[(-)] (-5,0)--(-3,0);
        \draw[very thick] (-5,0) --  (-3,0) node[midway, above]{$\frac{1}{N^n}$} node[below]{$u$};
        \draw[(-)] (-2,-1)--(-1.25,-1)node[midway, below]{$U_1$};
        %\draw[very thick] (0.92,0) -- (1.92,0);
        \draw[(-)] (0.5,-1) -- (1.75,-1)node[midway, below]{$U_2$};
        \draw[(-)] (3,0)--(6,0);
        \draw[very thick] (3,0) -- (6,0)node[midway, above]{$U'$};
        \draw[below,{[-]}]  (-3,-0.75) -- (3,-0.75) node[midway, above]{$C$};
        \draw[very thick]  (-3,-0.75) -- (3,-0.75);
    \end{tikzpicture}

    With this picture in mind lets take the infimum of the ratios.  The thickness of the $N$\hyphen{digit} expansion Cantor set is 
    $$\tau(K) = \inf_{u\in K}\frac{\ell(C)}{\ell(U)} = \min\left\{j,N-j-1\right\}$$
\end{example}  
%include examples of calculating newhouse thickness
%Cantor set Middle Third
%N Digit
%Comment on removing left interval doesn't work well.  Thin left interval versus Thick Right interval
%Also note that all self-similar Cantor sets in \R^1 have positive newhouse thickness
\section{The Gap Lemma}
\begin{lemma}[The Gap Lemma\cite{palis&takens}]
    Let $K_1, K_2, \subset \R$ be Cantor sets with thickness $\tau_1$ and  $\tau_2$.  If $\tau_1 \cdot \tau_2 >1,$  $K_1$ is not contained the gap of $K_2$, $K_2$ is not contained in the gap of $K_1$ then $K_1 \cap K_2 \neq \emptyset.$
\end{lemma}
% First we will assume for the sake of contradiction that the opposite is true.  These assumptions lead to claim, which under these assumptions must be true.  Then that claim leads to a contradiction which means our original assumption must be false.  
\begin{proof}
    Let $K_1, K_2$ be two Cantor sets with thickness $\tau_1, \tau_2$ respectively such that $K_1$ is not contained in the gap of $K_2$ and $K_2$ is not contained in the gap of $K_1$.  

    Assume for the sake of contradiction that $K_1 \cap K_2 = \emptyset$.  Consider the bounded gaps $U_1 \subset K_1^c$ and $U_2 \subset K_2^c$.  We call $(U_1, U_2)$ a gap-pair if $U_1$ contains exactly one boundary point of $U_2$ and $U_2$ contains exactly one point of $U_1$.  

    We want to use this technique to demonstrate that $K_1 \cap K_2 \neq \emptyset.$ Let $C_j^l, C_j^r$ denote the bridges of $K_j$ for $j = 1,2$.  Returning to our original assumptions, $\tau_1 \cdot \tau_2 > 1$ and therefore $$\frac{\ell(C_1)}{\ell(U_2)} \cdot \frac{\ell(C_2)}{\ell(U_1)} > 1.$$  
    %If $q \not\in K_1,$ then $q \in U_1'$, 
    The gap of $K_1$ where $\ell(U_1') < \ell(U_1)$ and $(U_1', U_2)$ is the gap pair we need.  
    
    From our construction, the right endpoint of $U_2$ is in $C_1^r$ or the left endpoint of $U_1$ is in $C_2^l$ or both.  By assumption we know that $K_1, K_2$ are not contained in the other's gaps. Therefore there exists some gap-pair $(U_1, U_2).$

    %In the case that $q \in U_2$ is the right endpoint, then $q \in K_1$ and $q \in K_2$ and we are done.  

    A quick picture can help show where these pieces are located:  This picture is not accurate because we are using this to derive a contradiction.
    
    \begin{tikzpicture}
        %First Cantor Set K_1
        \draw (-6.5,0) -- (6.5,0) ;
        \draw[(-)] (-5,0.25)--(-3,0.25) node[midway, above]{$U_1$};
        \draw[below,{[-]}]  (-3,-0.75) -- (3,-0.75) node[midway, above]{$C_1$};

        %second cantor set K_2
        \draw[{[-]}] (-6,0) --  (-4,0) node[midway, below]{$C_2$};
        \draw[(-)] (-4,-1)--(-1.25,-1)node[midway, below]{$U_2$};
        \draw (-6.5,-0.75) -- (6.5,-0.75) ;
    \end{tikzpicture}
    \begin{claim}
        If $\tau_1\tau_2 > 1$ then from the interval $U_1$ (or for that matter $U_2$) we can construct another sub-interval $U_1'$ such that $\ell(U_1') < \ell(U_1)$ (or similarly $U_2'$ such that $\ell(U_2')<\ell(U_2)).$
    \end{claim}
    %Proof of claim
    Notice that $(U_1',U_2)$ is still a gap-pair, as is $(U_1,U_2')$.
    
    Using this construction we can create a sequence of gap-pairs $(U_1^{(i)}, U_2^{(j)})$.  Notice that the sum is finite $$\sum_i^\infty \ell(U_1^{(i)}) < \infty,$$ and therefore $\ell(U_1^{(i)}) \to 0$ as $i \to \infty$.   From this we construction we have a sequence of gap-pairs such that as $i \to \infty$, $\ell(U_1^{(i)}) \to 0$ and similarly as $j \to \infty$, $\ell(U_2^{(j)}) \to 0$.  
    
    Without loss of generality, we can form a subsequence and use the same indexing for the gap pairs, $(U_1^{(i)}, U_2^{(i)})$.  
    
    This sequence of gap pairs have a non-empty intersection for all $i \in \N$.  

    Notice that by picking a sequence of points, $q_{i} \in U_1^{(i)}$ this forms a convergent subsequence $q_{i_k} \to q$.  Notice that $U_1^{(i)}$ 
    %= (a_i, b_i)$ 
    is not fully contained in the gap of $K_2$.  Moreover because these intervals are compact and nested, we know that %$a_{i_k} - b_{i_k} \to 0$ 
     $q$ which is contained in each $U_1^{(i)}$ is therefore in  $K_2$.  

    Because this construction is symmetric, the same argument applies to $q_{i} \in U_2^{(i)}$ and so $q \in K_2$.  Therefore the Cantor sets share at least one point, $q \in K_1 \cap K_2$.  

\end{proof}