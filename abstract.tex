\documentclass{amsart}
% maximum one page
\begin{document}

Often in data work, one may ask which patterns are possible to find, given a certain data set, or hypothetical relationship.  Mathematically a similar problem can be proposed.  Which patterns, finite or infinite, exist within another collection of sets?  A set is called universal in another set, when every subset of the larger set contains some scaled and translated copy of original.  Paul Erdos proposed a conjecture that no infinite set, is universal in the the collection of sets with positive measure.  This paper explores an analogous problem in a topological setting. Instead of sets with positive measure we investigate the collection of dense G-delta sets.  Any finite or countable set is found to be topologically universal.  Any set containing an interval cannot be topologically universal.  We also have the new result that any Cantor sets is not topologically universal.  Cantor sets, which contains no interval and are uncountably infinite, are not topologically universal in the collection of dense G-delta sets.  

%We  say that a set $E$ is  {\textit universal} in the collection of dense $G_{\delta}$ 
%sets if for all $G_{\delta}$ set,  
%we can  always find some affine copies of $E$ inside the set. 
%By an affine copy, we  mean sets of  the form $t+\lambda E$ 
%for some $t \in \mathbb{R}$ and $\lambda\neq 0$. 
%A natural question we have is that  is there a nowhere dense Cantor Set that is universal in the collection of dense $G_\delta$ sets? This is an exploration of an Erd\"{o}s conjecture in a topological setting. 




\end{document}